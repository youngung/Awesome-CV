\documentclass[10pt, letterpaper]{article}

% Packages:
\usepackage[
    ignoreheadfoot, % set margins without considering header and footer
    top=2 cm, % seperation between body and page edge from the top
    bottom=2 cm, % seperation between body and page edge from the bottom
    left=2 cm, % seperation between body and page edge from the left
    right=2 cm, % seperation between body and page edge from the right
    footskip=1.0 cm, % seperation between body and footer
    % showframe % for debugging
]{geometry} % for adjusting page geometry
\usepackage{titlesec} % for customizing section titles
\usepackage{tabularx} % for making tables with fixed width columns
\usepackage{array} % tabularx requires this
\usepackage[dvipsnames]{xcolor} % for coloring text
\definecolor{primaryColor}{RGB}{0, 0, 0} % define primary color
\usepackage{enumitem} % for customizing lists
\usepackage{fontawesome5} % for using icons
\usepackage{amsmath} % for math
\usepackage[
    pdftitle={John Doe's CV},
    pdfauthor={John Doe},
    pdfcreator={LaTeX with RenderCV},
    colorlinks=true,
    urlcolor=primaryColor
]{hyperref} % for links, metadata and bookmarks
\usepackage[pscoord]{eso-pic} % for floating text on the page
\usepackage{calc} % for calculating lengths
\usepackage{bookmark} % for bookmarks
\usepackage{lastpage} % for getting the total number of pages
\usepackage{changepage} % for one column entries (adjustwidth environment)
\usepackage{paracol} % for two and three column entries
\usepackage{ifthen} % for conditional statements
\usepackage{needspace} % for avoiding page brake right after the section title
\usepackage{iftex} % check if engine is pdflatex, xetex or luatex

% Ensure that generate pdf is machine readable/ATS parsable:
\ifPDFTeX
    \input{glyphtounicode}
    \pdfgentounicode=1
    \usepackage[T1]{fontenc}
    \usepackage[utf8]{inputenc}
    \usepackage{lmodern}
\fi

\usepackage{charter}

% Some settings:
\raggedright
\AtBeginEnvironment{adjustwidth}{\partopsep0pt} % remove space before adjustwidth environment
\pagestyle{empty} % no header or footer
\setcounter{secnumdepth}{0} % no section numbering
\setlength{\parindent}{0pt} % no indentation
\setlength{\topskip}{0pt} % no top skip
\setlength{\columnsep}{0.15cm} % set column seperation
\pagenumbering{gobble} % no page numbering

\titleformat{\section}{\needspace{4\baselineskip}\bfseries\large}{}{0pt}{}[\vspace{1pt}\titlerule]

\titlespacing{\section}{
    % left space:
    -1pt
}{
    % top space:
    0.3 cm
}{
    % bottom space:
    0.2 cm
} % section title spacing

\renewcommand\labelitemi{$\vcenter{\hbox{\small$\bullet$}}$} % custom bullet points
\newenvironment{highlights}{
    \begin{itemize}[
        topsep=0.10 cm,
        parsep=0.10 cm,
        partopsep=0pt,
        itemsep=0pt,
        leftmargin=0 cm + 10pt
    ]
}{
    \end{itemize}
} % new environment for highlights


\newenvironment{highlightsforbulletentries}{
    \begin{itemize}[
        topsep=0.10 cm,
        parsep=0.10 cm,
        partopsep=0pt,
        itemsep=0pt,
        leftmargin=10pt
    ]
}{
    \end{itemize}
} % new environment for highlights for bullet entries

\newenvironment{onecolentry}{
    \begin{adjustwidth}{
        0 cm + 0.00001 cm
    }{
        0 cm + 0.00001 cm
    }
}{
    \end{adjustwidth}
} % new environment for one column entries

\newenvironment{twocolentry}[2][]{
    \onecolentry
    \def\secondColumn{#2}
    \setcolumnwidth{\fill, 4.5 cm}
    \begin{paracol}{2}
}{
    \switchcolumn \raggedleft \secondColumn
    \end{paracol}
    \endonecolentry
} % new environment for two column entries

\newenvironment{threecolentry}[3][]{
    \onecolentry
    \def\thirdColumn{#3}
    \setcolumnwidth{, \fill, 4.5 cm}
    \begin{paracol}{3}
    {\raggedright #2} \switchcolumn
}{
    \switchcolumn \raggedleft \thirdColumn
    \end{paracol}
    \endonecolentry
} % new environment for three column entries

\newenvironment{header}{
    \setlength{\topsep}{0pt}\par\kern\topsep\centering\linespread{1.5}
}{
    \par\kern\topsep
} % new environment for the header

\newcommand{\placelastupdatedtext}{% \placetextbox{<horizontal pos>}{<vertical pos>}{<stuff>}
  \AddToShipoutPictureFG*{% Add <stuff> to current page foreground
    \put(
        \LenToUnit{\paperwidth-2 cm-0 cm+0.05cm},
        \LenToUnit{\paperheight-1.0 cm}
    ){\vtop{{\null}\makebox[0pt][c]{
        \small\color{gray}\textit{Last updated in September 2024}\hspace{\widthof{Last updated in September 2024}}
    }}}%
  }%
}%

% save the original href command in a new command:
\let\hrefWithoutArrow\href

% new command for external links:


\begin{document}
    \newcommand{\AND}{\unskip
        \cleaders\copy\ANDbox\hskip\wd\ANDbox
        \ignorespaces
    }
    \newsavebox\ANDbox
    \sbox\ANDbox{$|$}

    \begin{header}
        \fontsize{25 pt}{25 pt}\selectfont Youngung Jeong

        \vspace{5 pt}

        \normalsize
        \mbox{School of Materials Science and Engineering, Changwon National University}%
        \kern 5.0 pt%
        \AND%
        \kern 5.0 pt%
        \mbox{\hrefWithoutArrow{mailto:yjeong@changwon.ac.kr}{yjeong@changwon.ac.kr}}%
        \kern 5.0 pt%
        \AND%
        \kern 5.0 pt%
        \mbox{\hrefWithoutArrow{tel:+82-55-214-3694}{+82-55-214-3694}}%
        \kern 5.0 pt%
        \AND%
        \kern 5.0 pt%
        \mbox{\hrefWithoutArrow{https://youngung.github.io/}{youngung.github.io}}%
        \kern 5.0 pt%
        \AND%
        \kern 5.0 pt%
        \mbox{\hrefWithoutArrow{https://github.com/youngung}{github.com/youngung}}%
    \end{header}

    \vspace{5 pt - 0.3 cm}


%    \section{Welcome to RenderCV!}
%        \begin{onecolentry}
%            \href{https://rendercv.com}{RenderCV} is a LaTeX-based CV/resume version-control and maintenance app. It allows you to create a high-quality CV or resume as a PDF file from a YAML file, with \textbf{Markdown syntax support} and \textbf{complete control over the LaTeX code}.
%        \end{onecolentry}
%        \vspace{0.2 cm}
%        \begin{onecolentry}
%            The boilerplate content was inspired by \href{https://github.com/dnl-blkv/mcdowell-cv}{Gayle McDowell}.
%        \end{onecolentry}
%    \section{Quick Guide}
%    \begin{onecolentry}
%        \begin{highlightsforbulletentries}
%        \item Each section title is arbitrary and each section contains a list of entries.
%        \item There are 7 unique entry types: \textit{BulletEntry}, \textit{TextEntry}, \textit{EducationEntry}, \textit{ExperienceEntry}, \textit{NormalEntry}, \textit{PublicationEntry}, and \textit{OneLineEntry}.
%        \item Select a section title, pick an entry type, and start writing your section!
%        \item \href{https://docs.rendercv.com/user_guide/}{Here}, you can find a comprehensive user guide for RenderCV.
%        \end{highlightsforbulletentries}
%    \end{onecolentry}

    \section{Education}
        \begin{twocolentry}{Mar 2001 – Feb 2008}
            \textbf{Hanyang University}, BS, Department of Materials Science and Engineering
        \end{twocolentry}
        \vspace{0.10 cm}
        \begin{twocolentry}{Mar 2008 – Feb 2010}
            \textbf{POSTECH}, MS, Graduate Institute of Ferrous Technology (supervisor: F. Barlat)
        \end{twocolentry}
        \vspace{0.10 cm}
        \begin{twocolentry}{Mar 2008 – Feb 2010}
            \textbf{POSTECH}, PhD, Graduate Institute of Ferrous Technology (supervisor: F. Barlat)
        \end{twocolentry}
        \vspace{0.10 cm}
    \section{Experience}
        \begin{twocolentry}{Mar 2017 – Present}
            \textbf{Assistant, associate professor}, Changwon National University, ROK
        \end{twocolentry}
        \vspace{0.10 cm}
        \begin{twocolentry}{July 2024 – Aug 2024}
            \textbf{Short term visitor}, Los Alamos National Laboratory, NM, USA
        \end{twocolentry}
        \vspace{0.10 cm}
        \begin{twocolentry}{Feb 2022 – Feb 2024}
            \textbf{Guest Scientist (offsite)}, Los Alamos National Laboratory, NM, USA
        \end{twocolentry}
        \vspace{0.10 cm}
        \begin{twocolentry}{Dec 2016 – Feb 2017}
            \textbf{Postdoc}, POSTECH, ROK
        \end{twocolentry}
        \vspace{0.10 cm}
        \begin{twocolentry}{Mar 2016 – Nov 2016}
            \textbf{Research Scientist}, Clemson university, SC, USA
        \end{twocolentry}
        \vspace{0.10 cm}
        \begin{twocolentry}{Mar 2014 – Feb 2016}
            \textbf{Postdoc}, NIST, MD, USA
        \end{twocolentry}
        \vspace{0.10 cm}
        \begin{twocolentry}{Apr 2012 – Sep 2012}
            \textbf{Research Affiliate}, Los Alamos National Laboratory, NM, USA
        \end{twocolentry}
        \vspace{0.10 cm}
        \begin{twocolentry}{June 2011 – Dec 2011}
            \textbf{Guest Researcher}, NIST, MD, USA
        \end{twocolentry}
        \vspace{0.10 cm}

%        \begin{twocolentry}{
%            June 2003 – Aug 2003
%        }
%            \textbf{Software Engineer Intern}, Microsoft -- Redmond, WA\end{twocolentry}

%        \vspace{0.10 cm}
%        \begin{onecolentry}
%            \begin{highlights}
%                \item Designed a UI for the VS open file switcher (Ctrl-Tab) and extended it to tool windows
%                \item Created a service to provide gradient across VS and VS add-ins, optimizing its performance via caching
%                \item Built an app to compute the similarity of all methods in a codebase, reducing the time from $\mathcal{O}(n^2)$ to $\mathcal{O}(n \log n)$
%                \item Created a test case generation tool that creates random XML docs from XML Schema
%                \item Automated the extraction and processing of large datasets from legacy systems using SQL and Perl scripts
%            \end{highlights}
%        \end{onecolentry}

    \section{Selected recent publications}
        \begin{samepage}
            \begin{twocolentry}{2024}
                \begin{highlights}
                \item\href{https://doi.org/10.1016/j.jmrt.2024.11.043}{A critical discussion of elasto‑visco‑plastic self‑consistent (EVPSC) models}
                \end{highlights}
            \end{twocolentry}
            \begin{onecolentry}
                \mbox{B. Jeon}, \mbox{\textbf{\textit{Y. Jeong}}}, \mbox{C. N. Tomé}
            \end{onecolentry}
            \begin{onecolentry}
                Journal of Materials Research Technology
            \end{onecolentry}
            \vspace{0.10 cm}

            \begin{twocolentry}{2024}
                \begin{highlights}
                \item\href{https://doi.org/10.1016/j.ijplas.2024.104098}{Direct application of elasto‑viscoplastic self‑consistent crystal plasticity model to U‑draw bending and springback of dual‑phase high strength steel}
                \end{highlights}
            \end{twocolentry}
            \begin{onecolentry}
                \mbox{B. Jeon}, \mbox{S.-Y. Lee}, \mbox{J. Lee}, \mbox{\textbf{\textit{Y. Jeong}}}
            \end{onecolentry}
            \begin{onecolentry}
                \mbox{International Journal of Plasticity}
            \end{onecolentry}
            \vspace{0.10 cm}

            \begin{twocolentry}{2023}
                \begin{highlights}
                \item\href{https://doi.org/10.1016/j.ijmecsci.2022.107796}{A crystal plasticity finite element analysis on the effect of prestrain on springback}
                \end{highlights}
            \end{twocolentry}
            \begin{onecolentry}
                \mbox{M. Joo}, \mbox{M. -S. Wi}, \mbox{S.-Y. Yoon}, \mbox{S.-Y. Lee}, \mbox{F. Barlat}, \mbox{C. N. Tomé}, \mbox{B. Jeon}, \mbox{C. N. Tomé}, \mbox{\textbf{\textit{Y. Jeong}}},
             \end{onecolentry}
             \begin{onecolentry}
               \mbox{International Journal of Mechanical Sciences}
             \end{onecolentry}
            \vspace{0.10 cm}

            \begin{twocolentry}{2021}
                \begin{highlights}
                \item\href{https://doi.org/10.1016/j.ijplas.2021.103110}{Finite element analysis using an incremental elasto‑visco‑plastic self‑consistent polycrystal model: FE simulations on Zr and low‑carbon steel subjected to bending, stress‑relaxation, and unloading}
                \end{highlights}
            \end{twocolentry}
            \begin{onecolentry}
                \mbox{\textbf{\textit{Y. Jeong}}}, \mbox{B. Jeon}, \mbox{C. N. Tomé}
            \end{onecolentry}
            \begin{onecolentry}
                \mbox{International Journal of Plasticity}
            \end{onecolentry}
            \vspace{0.10 cm}

            \begin{twocolentry}{2020}
                \begin{highlights}
                \item\href{https://doi.org/10.1016/j.ijplas.2020.102812}{An efficient elasto‑visco‑plastic self‑consistent formulation: Application to steel subjected to loading path changes}
                \end{highlights}
            \end{twocolentry}
            \begin{onecolentry}
                \mbox{\textbf{\textit{Y. Jeong}}}, \mbox{C. N. Tomé}
            \end{onecolentry}
            \begin{onecolentry}
                \mbox{ International Journal of Plasticity}
            \end{onecolentry}
            \vspace{0.10 cm}

        \end{samepage}



    \section{GitHub repositories}
        \begin{twocolentry}{\href{https://github.com/youngung/evpsc}{github.com/youngung/evpsc}}\textbf{Elasto-visco-plastic self-consistent model} (private repo)
        \end{twocolentry}
        \vspace{0.10 cm}
        \begin{onecolentry}
            \begin{highlights}
                \item Extended visco-plastic self-consistent model to account for elasticity
                \item Offer both stand-alone and FE-interface simulation capabilities
                \item Written primarily in Fortran with Python and shell scripts.
                \item Finite element simulation via user material interface of Abaqus/standard solver
            \end{highlights}
        \end{onecolentry}

        \vspace{0.2 cm}
        \begin{twocolentry}{\href{https://github.com/youngung/texture3}{github.com/youngung/texture3}}
            \textbf{In-house Python scripts for texture analysis}
        \end{twocolentry}
        \vspace{0.10 cm}
        \begin{onecolentry}
            \begin{highlights}
                \item Contoured pole figures for discrete orientations used in VPSC-like crystal plasticity codes
                \item Written entirely in Python with open-sourced libraries including matplotlib, NumPy and SciPy
            \end{highlights}
        \end{onecolentry}

    \section{Skills}
        \begin{onecolentry}
            Fortran, Python (NumPy, SciPy, and matplotlib), Matlab, shell scripts, Git
        \end{onecolentry}
\end{document}